\section{Development Method} \label{sec:DevelopmentMethod}
For this project, an approach similar to that adopted by Bennett et al. (2010,
p. 77) \nocite{bennett2010object} regarding methodology will be employed, where
no specific named methodology is espoused, but concepts of object-oriented
analysis and design shall be applied, in an iterative and incremental fashion,
using UML.

\subsection{Universal Modelling Language (UML)}
\label{sec:Introduction.methodology.uml}

UML is a modelling language created with the intention of providing system
architects, software engineers and developers with a common set of modelling
tools with a defined syntax which would help them better analyse, design and
implement software-based systems, and to model business and similar processes
(\cite[][p.~43]{omg2015uml}). It defines several constructs which will be
employed throughout this report in order to model the specifications of the
system, such as:
\begin{itemize}
  \item
    Use Case diagrams;

  \item
    Activity diagrams;

  \item
    Class diagrams;
    
  \item
    Sequencec diagrams;
\end{itemize}

Some or all diagrams will be drawn with a rectangular frame around them, with a
five-sided polygon containing its header at the top, as per the specifications
for UML 2.0 onwards (\cite[][p.~125]{bennett2010object}). This is done for ease
of reference, in order for the reader to better identify which aspect of the
problem domain is being modelled.


\subsection{Requirements Modelling} \label{sec:DevelopmentMethod.requirementsModelling}
Benett et al. (2010,p. 140-142) \nocite{bennett2010object} categorises
requirements in three types:

\begin{description} \item[Functional Requirements]
    The system's functionality -- what it is expected to do.
    
  \item[Non-functional Requirements]
    How well the system delivers its functionality. These requirements are
    related to the performance, scalability, availability, recovery time,
    security, and others.

  \item[Usability Requirements]
    These relate to how effectively, efficiently and satisfactorily users can
    achieve their goals in the existing system. User interfaces can play a big
    part in meeting these requirements.
\end{description}

The initial implementation iterations will be focused more on the functional
and usability requirements, paying some attention as well to specific
non-functional requirements such as performance and security.

Use case diagrams are an UML construct which were developed by Jacobson et al.
(1992, cited \cite[][p.~154]{bennett2010object}). They will be used, along with
other tools, to model functional requirements.
