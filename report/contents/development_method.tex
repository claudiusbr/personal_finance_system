\section{Development Method} \label{sec:DevelopmentMethod}
For this project, an approach similar to that adopted by Bennett et al.
(\citeyear[][p.~77]{bennett2010object}) regarding methodology has been
employed, where no specific named methodology is espoused, but concepts of
object-oriented analysis and design were applied, in an iterative and
incremental fashion, using UML. More details about which concepts were used and
the methodologies which originated them can be found in the following
subsections.

\subsection{The use of Universal Modelling Language (UML) constructs}
\label{sec:Introduction.methodology.uml}

UML is a modelling language created with the intention of providing system
architects, software engineers and developers with a common set of modelling
tools, with a defined syntax, which would help them better analyse and design
software-based systems, and to model business and similar processes
(\cite[][p.~43]{omg2015uml}). It defines several constructs which have been
employed throughout this report in order to model the specifications of the
system, such as:
\begin{description}
  \item[Use Case diagrams]
    As a useful, high level tool to document users' requirements
    (\cite[][p.~138]{bennett2010object}), use case diagrams have been used to
    develop the requirements model of the system.

  \item[Activity diagrams]

  \item[Class diagrams]
    
  \item[Sequence diagrams]
\end{description}


\subsection{Requirements Capture Methods} \label{sec:DevelopmentMethod.RequirementsCapture}
Due to the nature of the system being for personal rather than commercial use
-- that is, by individuals rather than business entities -- the usual fact
finding techniques do not apply specifically well. However, the closest match
identified to the techniques utilised has been with \emph{`Knowledge
Acquisition'}. This relates to the process of capturing knowledge from an
expert (\cite[][p.~150]{bennett2010object}). In this particular case, though
perhaps not qualifying as an expert, the author's qualification and experience
in accounting and bookkeeping was used to capture the main requirements.

\subsection{Analysis and Design} 
As described by Bennett et al (\citeyear[][p.~348]{bennett2010object}), ``in
projects that follow an iterative lifecycle, design is not such a clear-cut
stage, but rather an activity that will be carried out on the evolving model of
the system''. Seeing that the development method being followed in this project
is based on an iterative approach, it was decided that the analysis and design
of it would be done concurrently.

\subsubsection{Analysis and Design Patterns}
Fowler (\citeyear[][Section~1.3]{fowler1997analysis}) defines a pattern as ``an
idea that has been useful in one context and will probably be useful in
others''. This project will therefore attempt to utilise patterns where
appropriate in order to prove this concept, and as an attempt to make use of
the experience already acquired in the domain (or domains) in question. As
emphasised by Bennett et al. (\citeyear[][p.~252]{bennett2010object}), ``a
pattern is useful when it captures the essence of a problem and a possible
solution, without being too prescriptive''. So it may be the case that some of
the patterns will be modified where necessary to optimally solve a problem.

The concept of \emph{domain} here is being used, as defined by Evans
(\citeyear[][p.~2]{evans2004domain}), as the ``activity or interest of its
user'' -- the ``subject area to which the user applies the program''.

Analysis patterns will be used ``when trying to understand the problem'' domain
(\cite[][Section~1.1]{fowler1997analysis}). Essentially, an analysis pattern
consists of a structure of classes and associations which occurs often in many
modelling situations related to specific domains
(\cite[][p.~254]{bennett2010object}).
