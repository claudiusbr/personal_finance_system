\section{Reflections} \label{sec:Reflections}

\subsection{Use Case Templates} \label{sec:Reflections.UseCaseTemplates}
Originally, no template was used to document the use cases. The intention was
to provide better ones at a later iteration, perhaps by researching the ones
mentioned by Bennett et al. (\citeyear[][p.~157]{bennett2010object}), but
unfortunately there was not enough time, so the little there was had to be
dedicated to the software itself.

\subsection{Nested iterations in Analysis and Design stage}
The Analysis and Design stage of the first iteration was delayed due to
multiple `trial and errors' within it. This caused a reflection on whether the
development method was truly iterative, or whether or not it was more similar
to the Waterfall model.

Still, in the author's opinion, spending more time within the analysis and
design stage were very helpful in implementing \emph{SOLID} classes, and as a
result decreased the negative impacts that any refactoring in the code base
would have caused, were it not to have been done as such. In previous
iterations of similar projects by the author, but where the modelling normally
done in the analysis/design stages were neglected, classes ended up with too
many responsibilities and hard coded dependencies, which made any refactoring
very challenging.


\subsection{Trade-off between less code duplication and more sub-package independence} \label{sec:Reflections.tradeoffs}
In order to keep sub-packages independent, as described in sub-section
\ref{sec:Implementation.Presentation}, a decision had to be made about
increasing code duplication. As can be seen in the current implementation, both
\texttt{PlayMediator} and \texttt{SwingMediator} are supposed to declare the
same method signatures, so that there is no conflict when switching between
implementations at run time. In order for this to happen, each interface had to
be manually typed into its package, and the same will have to happen for any
further extensions to the code. An alternative to this would be to allow the
sub-packages containing the implementations to see the super-package, and
declare the methods once in the \texttt{PresentationMediator} interface, but
this would have meant sacrificing the concept of sub-package independence
already mentioned.

\subsection{Layering vs Manual Dependency Injection}
One of the original (implicit) goals of this project was to have a hierarchy
system, where each sub-package would not depend super package, but super
packages could depend on sub ones. That is, a highly specialised
\texttt{presentation.swing.frames} package would not depend on elements of the
\texttt{presentation} package, but the \texttt{swing.frames} package would
declare interfaces which would then be implemented by \texttt{presentation}.
What had not been taken into consideration, however, is how the fact that the
super package having to implement the interface would make it difficult to
truly implement dependency injection: the idea was to have the
\texttt{InteractionMediator} implement the lower package's interfaces, so that
it could be passed to the classes of the specialised package. But this would be
a problem when the interface is too specialised. Therefore, a compromise had to
be made and the interfaces had to be made more generic.

\subsection{Not implemented due to time constraints} \label{sec:Reflections.TimeConstraints}
The following were not implemented due to time constraints.

\subsubsection{Better exception handling and communication with the user}
The \texttt{PersistenceMediator} class should be handling exceptions which
might be thrown by \texttt{PersistenceBridge}'s every time the first calls the
latter's methods. The intention of allowing the Mediator to catch exceptions
was because this could then be passed to the user as informative messages, or
be handled internally depending on the nature of the exception (e.g.,
connection exceptions would have to be handled internally, and exceptions
related to user input should be passed to the user). Therefore, some exceptions
should be handled by the \texttt{PersistenceBridge}, and others by the
mediator.

A feature to communicate with the end user was being implemented as well by
means of the \texttt{presentation.swing.Messenger} object. Unfortunately, there
were bugs in the code which did not allow it to work properly, and a solution
could not be found in time for the deadline, which meant this feature was not
implemented in the MVP.

\subsubsection{Validation} \label{sec:Reflections.TimeConstraints.Validation} 
A lot of thought has been put into where validation should happen. For example,
the constraints of \texttt{Transaction}, \texttt{Category} and \texttt{Entry}
which were used to enforce \emph{double-entry} could have been implemented at
database or application levels, or both. In the current implementation,
however, this constraint is only enforced at business logic layer, and in the
\texttt{PersistenceBridge.createEntrySet} method of the persistence layer.

This means, unfortunately, that the user is still able to bypass double entry by
directly accessing the database, especially since the data is being stored
unencrypted. It should be possible to design the application so that the only
possible access point to the data will be the application, and that constraints
will be enforced at every level, but this will have to be left to a future
iteration, which will fall outside of the scope of this report.

Regarding user input validation, the current implementation left a lot of room
for improvement in this area. As the current implementation is a desktop app,
the risk mainly consists of bad input due to user misunderstandings of the
interface, or errors, and not so much of malicious intent. However, it would
still be in line with best practices to have an interface that is as ``immune''
to user-errors as possible. Special characters are not yet being properly
handled, but which should be fixed on a future iteration outside of the scope
of this report.

Another feature which relates to validation is checking for duplicate entries.
As it stands, it is possible to upload the same statement, or make the same
manual entry, twice. This is another feature which will have to be implemented
at a possible future iteration, outside of the scope of this report.


\subsubsection{Implementation of the Strategy Design Pattern for Parser}
One of the original intentions of the author was to utilise an implementation
of the \emph{Strategy} pattern when loading the user's bank statements into the
system. The current implementation uploads data from CSV files, but making use
of the Strategy pattern, which allows for different implementations of an
algorithm to exist, and for the right one to be chosen while the application is
\emph{running} (\cite[][Ch.~8,~Location~3152]{nikolov2016scala}), would allow
for other formats to be used too. JSON and XML formats come to mind, especially
if a version of the application could be made which would allow for it to
communicate with a banking system's API -- API's (especially RESTful) tend to
favour these two formats. This did not happen in this iteration, unfortunately.

\subsubsection{The Over-simplicity of the budgeting feature}
The current implementation of the budget feature would not work well if
the user has overspent: it will simply show an increase the projected income to
match the user's expenditure, which would not be a desirable trait in a
commercial expenditure analyser -- this implies that the application is telling
the user to ``earn more'', rather than spend less. There are better ways to
implement this feature, such as decreasing some of the expenditure when
budgeting, but the time constraints once again prevented this from being built.


\subsubsection{Implementation of Presentation Layer using only Functional Paradigm}
After the first iteration of the presentation layer, it was noticed that
perhaps it could be fully implemented in a form more close to that of the
functional paradigm. That is, make it a point to not use \texttt{var}'s, and
only \texttt{val}'s for members (also not change state using Scala Swing's
classes natural mutable fields, such as location, visible, etc -- the effects
from these could be replicated by copying the values of the instances into new
ones). This could have been achieved if the full \texttt{presentation.swing}
package had been implemented from the start with this in mind: have an
interaction mediator within the package, and then make more use of double
dispatch, familiar to the \emph{Visitor Pattern}
(\cite[][Ch.~8,~Location~3943]{nikolov2016scala}). Then, for the actual flow of
the application, a strategy similar to that used by Felleisen et al.
(\citeyear[][Ch.~5]{felleisen2013realm}), where each action would trigger a
function which changes the state of the application, and then the GUI
displaying the new state would be passed recursively to to the main function,
ensures that the immutability of the functional paradigm is maintained
throughout the presentation layer.

Unfortunately, time constraints were once again too tight for this to be fully
implemented.

\subsubsection{Decimal Precision}
The application was implemented using \texttt{Double} as the data type.
Unfortunately, after testing it with a larger dataset, it was noticed that this
was affecting the precision of the data. Due to time constraints it was not
possible to switch all the necessary classes to \texttt{BigDecimal}, or another
similar implementation which would allow for decimal precision to be achieved.
This should happen in a future iteration, however.

\subsubsection{Aesthetics}
For the current implementation, the aesthetics of the front end were heavily
neglected. During development of this version of the product, although some
attention was being given to achieve symmetry -- one of the aspects of
classical aesthetics  -- this was only possible in the
limited time because \emph{Scala Swing} makes symmetry almost intrinsic to
the organisation of its \texttt{BoxPanel} class. All other aspects of classical
and expressive aesthetics (Lavie and Tractinsky, 2004,
\cite[cited][p.~102]{benyon2005designing}) had to be left for a future
iteration.
