\section{Reflections} \label{sec:Reflections}

\subsection{Expert Systems} \label{sec:Reflections.ExpertSystems}
Originally, the author did not know about expert systems when the idea for this
project was conceived. However, during the literature search and review, the
idea for these systems was found, and many of the patterns of what an expert
system does and what this system is supposed to do were identified to be
similar. This led to the conclusion that the project has the potential to
become an expert system, even if just with a budgeting tool. The expert
knowledge being provided by it for its first iteration includes:
\begin{itemize}
  \item
    Double entry bookkeeping;

  \item
    Budgeting by category.
\end{itemize}

It achieves the above by separating the inference engine, which is the tool
responsible for knowing how to apply double entry to transactions, from its
knowledge base which is the information input by the user -- for example, if
the user tells the system a manual entry is income, the system will know to
debit the cash book, and credit the category in question.

Brown et al. (\citeyear{brown1990expert}) declare that the heuristics used by a
financial planning system can be interpreted as a ``rule of thumb'' to be
applied to a problem which will normally result in a correct solution for it.
In the same article it is also stated that ``an expert system is most commonly
and most effectively used as an advisor to a human decision maker''. I this is
considered as the measure by which classify an expert system, then the
budgeting tool alone would place this system into it.

\subsection{Use Case Templates} \label{sec:Reflections.UseCaseTemplates}
Originally, no template was used to document the use cases. The intention was
to provide better ones at a later iteration, perhaps by researching the ones
mentioned by Bennett et al. (\citeyear[][p.~157]{bennett2010object}), but
unfortunately there was not enough time, so the little there was had to be
dedicated to the software itself.
