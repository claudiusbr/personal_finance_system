\section{Reflections} \label{sec:Reflections}

\subsection{Expert Systems} \label{sec:Reflections.ExpertSystems}
Originally, the author did not know about expert systems when the idea for this
project was conceived. However this concept was found during the literature
search and review, and many of the patterns of what an expert system does and
what this system is supposed to do were identified to be similar. This led to
the conclusion that the project has the potential to become an expert system,
even if just with a budgeting tool. The expert knowledge being provided by it
for its first iteration includes:
\begin{itemize}
  \item
    Double entry bookkeeping;

  \item
    Budgeting by category.
\end{itemize}

It achieves the above by separating the inference engine, which is the tool
responsible for knowing how to apply double entry to transactions, from its
knowledge base which is the information input by the user -- for example, if
the user tells the system a manual entry is income, the system will know to
debit the cash book, and credit the category in question.

Brown et al. (\citeyear{brown1990expert}) declare that the heuristics used by a
financial planning system can be interpreted as a ``rule of thumb'' to be
applied to a problem which will normally result in a correct solution for it.
In the same article it is also stated that ``an expert system is most commonly
and most effectively used as an advisor to a human decision maker''. I this is
considered as the measure by which classify an expert system, then the
budgeting tool alone might qualify this system as one.

\subsection{Use Case Templates} \label{sec:Reflections.UseCaseTemplates}
Originally, no template was used to document the use cases. The intention was
to provide better ones at a later iteration, perhaps by researching the ones
mentioned by Bennett et al. (\citeyear[][p.~157]{bennett2010object}), but
unfortunately there was not enough time, so the little there was had to be
dedicated to the software itself.

\subsection{Iterations within the analysis and design iteration}
The analysis and design stages of the first iteration was delayed due to
multiple `trials and errors' within it. Appendices \ref{appendix2} and
\ref{appendix3} show a examples of different models which had been incorporated
into the final design, but which were later changed even before the
implementation started -- that is, even before any coding was done. This caused
a reflection on whether the development method was truly iterative, and whether
or not it was more similar to the Waterfall model.

\subsection{Dependency Injection} \label{sec:Reflections.DependencyInjection}
Very often during the implementation it was felt that better dependency
injection could be achieved. In classes such as those found in the
\texttt{TransactionsValidators.scala} file there was constant hard-coded
dependency to the \texttt{validation} package. In this instance, a framework
such as \emph{Spring} or \emph{Guice} would have been useful, but unfortunately
time constraints made it unlikely for the author to implement these into the
project. As a result, there is more tight coupling than there could be, but
wherever possible an effort has been made to pass the dependencies as
constructor or method parameters, so as to facilitate testing, among other
things.

\subsection{Validation} \label{sec:Reflections.Validation} 
\textbf{TODO: see if this is still the case at the end of the project}
A lot of thought has been put into where validation should happen. For example,
the constraints of \texttt{Transaction}, \texttt{Category} and \texttt{Entry}
which were used to enforce \emph{double-entry} could have been implemented at
database or application levels, or both. Initially they were implemented in the
business logic.

