\section{Testing} \label{sec:Testing}

As can be seen in the test suites implemented, the testing for this project was
an attempt at Behaviour-Driven Development, which is a form of Test-Driven
Development which focuses on trying to write tests by including descriptions of
the expected behaviour, which is also an attempt to make the code more readable
(\cite[][Ch.~1]{wynne2017cucumber}). Scala's infix notation also make it easy
to write more readable tests, due to its infix notation, and the aid of the
\texttt{FlatSpec} class and \texttt{Matchers} trait.

Most of the automated testing consist of integration tests, since more groups
of objects are being tested together but not necessarily the whole application.
This was chosen where it felt it would be beneficial to test how well the more
heavily integrated objects interacted. Whenever it was appropriate, mocks of
other objects were used, as can be seen in the \texttt{StringClassifierTester}
class. The \emph{Mockito} libraries were used for object mocking, as they work
well with \texttt{ScalaTest}, with the aid of Scala's \texttt{MockitoSugar}
trait, to aid with the syntax (\cite[][pp.~102-106]{hinojosa2013testing}).


The automated tests cover mostly the persistence and business logic layers, and
unfortunately not much has been done with the presentation layer due to time
constraints. For the persistence layer, a test schema was created, and it needs
to be loaded into a \emph{MySQL} database before it can be used. The
specifications for these need to be entered into \texttt{.properties} files
before running the tests, otherwise they will fail. Each time they run, any
changes made to the test database will be overwritten by the persistence
helper, which ensures consistency. The only reason this was done with
\emph{MySQL} for the current implementation is that the database is local,
therefore performance is not significantly affected. However, for future
iterations a portable database would be more suitable for persistence layer
testing.
