\section{Testing} \label{sec:Testing}

As can be seen in the test suites implemented, the testing for this project was
an attempt at Behaviour-Driven Development (BDD), which is a form of
Test-Driven Development that focuses on trying to write tests by including
descriptions of the expected behaviour, which is also an attempt to make the
code more readable (\cite[][Ch.~1]{wynne2017cucumber}). The main library used
for testing was \texttt{ScalaTest}, ``an extensive BDD suite with numerous
built-in specs'' (\cite[][p.~21]{hinojosa2013testing}). Some of Scala's native
features also make it easy to write more readable tests, such as its infix
notation, which work well with the members of the \texttt{FlatSpec} class and
\texttt{Matchers} trait.

Most of the automated testing consist of integration tests, since more groups
of objects are being tested together but not necessarily the whole application.
This was chosen where it felt it would be beneficial to test how well the more
heavily integrated objects interacted. Whenever it was appropriate, mocks of
other objects were used, as can be seen in the \texttt{StringClassifierTester}
class. The \emph{Mockito} libraries were used for object mocking, as they work
well with \texttt{ScalaTest}, with the aid of Scala's \texttt{MockitoSugar}
trait, to aid with the syntax (\cite[][pp.~102-106]{hinojosa2013testing}).

The automated tests cover mostly the persistence and business logic layers, but
unfortunately not much was done with the presentation layer due to time
constraints. For the persistence layer, a test database schema was created, and
it needs to be loaded into a \emph{MySQL} database before it can be used. The
specifications for these need to be entered into \texttt{.properties} files
before running the tests, otherwise they will fail. Each time they run, any
changes made to the test database will be overwritten by the persistence
helper, which ensures consistency. The only reason this was done with
\emph{MySQL} for the current implementation is that the database is local,
therefore performance is not significantly affected. However, for future
iterations a portable database would be more suitable for persistence layer
testing.

Since TDD was used, in the vast majority of cases the tests were implemented
before the classes being tested were written. The classes were then written and
refactored until the tests were satisfied. In some cases, further tests were
written where it was noticed that more specific behaviours were required of
particular classes. After the tests were written, they behaved like a harness
which allowed for any further necessary changes to be made with confidence, and
helped to speed up the implementation phase
(\cite[][pp.~x-xi]{hinojosa2013testing}). The result of running the tests at
the time of writing can be seen in \hyperref[appendix4]{Appendix IV}.


Whenever `on-the-fly', manual tests were needed, these were done using the
Simple Build Tool (SBT) console feature. This feature is a very interesting
combination of Scala's Read-Eval-Print-Loop (REPL) with the build tool's
ability to quickly integrate all necessary dependencies into a REPL console
session (\cite[][p.~1]{hinojosa2013testing}). The advantage of testing in this
manner was that it was possible to run a session with compiled code incomplete
classes and test different aspects of their behaviour which did not seem fit
for automated testing.
