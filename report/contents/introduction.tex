\section{Introduction} \label{sec:Introduction}

A system could be summarised roughly as a solution to one or more problems. One
of the first steps in order to build this kind of solution is to try to
understand the problem -- that is, try to map the requirements of the software.
Vaasen et al. (2009 \cite[cited][p.~8]{Boczko:2012:IAI:2331376}) suggests that
an accounting information system's main purpose is to provide information to
internal and external stakeholders. Although this refers to accounting systems
for businesses, it could be argued that this same definition can be employed to
define personal finance systems -- except that, in this case, the main
stakeholder would be the individual using the system (that is, the user). In
fact, one of the most widely known accounting systems available in the market,
Quicken\texttrademark, was conceived around the idea that there should be more
efficient and less tedious ways to organise one's personal financial
information than doing it manually (\cite{quicken2017about}). This project has
been developed based on similar ideas.

It seems fair to infer that nowadays most of a user's financial transactions
happen in ways that can be listed electronically (usually via their bank or
credit card statements) -- a study by Payments UK
(\citeyear{paymentsUK2017summary}), for example, indicates that there has been
a rise in debit card payments over the past few years, and that the volumes of
this type of transaction is likely to be higher than that of cash payments by
the year 2021. Therefore, an assumption has been made that the users will
require means of uploading a list of their financial transactions into the
system.

The system created for this project intends to do just this. Its main feature,
however, will be to allow the user to categorise expenditure based on patterns
in the entries' descriptions. Aside from this, there will also be a feature to
allow the user to view summaries of the income and expenditure over a period of
time, and another one to generate budget forecasts for future periods based on
the financial information already entered.

This report documents the work of the project. Each chapter delineates a
specific aspect of the development life cycle, which is in line with the
development process described in the following paragraphs. Chapter
\ref{sec:Requirements} outlines the identified requirements which were used as
motivation for the system to be developed; the contents of Chapter
\ref{sec:AnalysisAndDesign} shows further analysis and concomitant design of
the system and the solutions it brings; Chapter \ref{sec:Implementation}
outlines select aspects of the implementation stage which serve to emphasise
the techniques used to implement the designed logic, or highlights areas where
it was felt it was necessary to implement something slightly different that
what was designed.

The system developed for this project has been modelled after the principle of
\emph{double entry bookkeeping}, from the accounting domain, which states that
``money is never created or destroyed -- it merely moves from one account to
another'' (\cite[][Section 6.2]{fowler1997analysis}). More specifically, double
entry is the principle which ensures that every transaction always affects two
accounts, one being credited (Out) and one debited (In). An account, for the
scope of this project, refers either to a category created by the user, or to
the user's \emph{cash book} -- the contents of their bank account plus any
manual entry which they make. In bookkeeping, each account can be classified as
\emph{asset, liability, income} or \emph{expenditure}. Whether the account
increases or decreases will depend on which of these categories it falls under:
\emph{debits} will increase \emph{assets} and \emph{expense} accounts, and
\emph{credits} will increase \emph{liability, capital} or \emph{income}
accounts (\cite[][pp.~18-19]{wood2004book}).


Regarding the development method, an approach similar to that adopted by
Bennett et al.  (\citeyear[][p.~77]{bennett2010object}) regarding software
analysis an design has been employed, where no specific named methodology is
espoused, but concepts of object-oriented analysis and design were applied, in
an iterative and incremental fashion, using UML. More details about which
concepts were used and the methodologies which originated them can be found in
the following subsections.


The remaining definitions from \hyperref[appendix1]{Appendix I}, including
those of functional and non-functional requirements, will be employed when
trying to classify the requirements and model the problem domain. The initial
iterations will be focused more on the functional and usability requirements,
paying some attention as well to specific non-functional requirements such as
performance and security.


\subsection{Requirements Capture Methods} \label{sec:DevelopmentMethod.RequirementsCapture}
The closest match identified to the techniques utilised for requirements
capture for this project was \emph{`Knowledge Acquisition'}, this relates to
the process of capturing knowledge from an expert
(\cite[][p.~150]{bennett2010object}). In this particular case, though perhaps
not qualifying as an expert, the author's own qualifications and experience in
accounting and bookkeeping were used to generate the ideas from which the
requirements were extracted.

Differently from the original plan, and more in line with the \emph{incremental
model} of development exemplified by \cite[][pp.~120-124]{dawson2009projects},
the requirements were gathered in full before any actual development started.

\subsection{Analysis and Design} 
As described by Bennett et al (\citeyear[][p.~348]{bennett2010object}), ``in
projects that follow an iterative life cycle, design is not such a clear-cut
stage, but rather an activity that will be carried out on the evolving model of
the system''. Seeing that the development method being followed in this project
is based on what should be an iterative approach, it was decided that the
analysis and design of it would be done concurrently.

\subsubsection{Analysis and Design Patterns}
Fowler (\citeyear[][Section~1.3]{fowler1997analysis}) defines a pattern as ``an
idea that has been useful in one context and will probably be useful in
others''. This project will therefore attempt to utilise patterns where
appropriate in order to prove this concept, and as an attempt to make use of
the experience already acquired in the domain (or domains) in question. As
emphasised by Bennett et al. (\citeyear[][p.~252]{bennett2010object}), ``a
pattern is useful when it captures the essence of a problem and a possible
solution, without being too prescriptive''. So it may be the case that some of
the patterns will be modified where necessary to optimally solve a problem.

The term \emph{domain} is being employed, as defined by Evans
(\citeyear[][p.~2]{evans2004domain}), to define the ``activity or interest of
its user'' -- the ``subject area to which the user applies the program''.  
So, analysis patterns will be used ``when trying to understand the problem'' domain
(\cite[][Section~1.1]{fowler1997analysis}). Essentially, an analysis pattern
consists of a structure of classes and associations which occurs often in many
modelling situations related to specific domains
(\cite[][p.~254]{bennett2010object}).
