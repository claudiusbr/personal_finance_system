\section{Introduction} \label{sec:Introduction}

Vaasen et al. (2009 \cite[cited][p.~8]{Boczko:2012:IAI:2331376}) suggests that
an accounting information system's main purpose is to provide information to
internal and external stakeholders. Although this refers to accounting systems
for business, it could be argued that the same concept could be applied for
personal finance systems -- except that, in this case, the main stakeholder
would be the individual using the system (that is, the user). In fact, one of
the most widely known accounting systems available in the market,
Quicken\texttrademark, was conceived around the idea that there should be more
efficient and less tedious ways to organise one's personal financial
information than doing it manually (\cite{quicken2017about}). This project has
been developed based on these ideas.

The system has been developed using the principle of \emph{double entry
bookkeeping}, which states that every transaction should affect two accounts,
one being credited (Out) and one debited (In). An account, for the scope of
this project, either refers to a category created by the user, or to the user's
\emph{cash book} -- the contents of their bank account plus any manual entry
which they make.  In bookkeeping, each account can be classified as an
\emph{asset, liability, income} or \emph{expenditure}. Whether the account
increases or decreases will depend on which of these categories it falls under:
\emph{debits} will increase \emph{assets} and \emph{expense} accounts, and
\emph{credits} will increase \emph{liability, capital} or \emph{income}
acccounts (\cite[][pp.~18-19]{wood2004book}).

This report documents the work of the project. Each chapter delineates a
specific aspect of the development lifecycle, which is in line with the
development process listed in \ref{sec:DevelopmentMethod}. Chapter
\ref{sec:Requirements} identifies the identified requirements which were used
as motivation for the system to be developed.

Analysis of requirements has been incorporated in the design section (Chapter
\ref{sec:Design}). This is also where the first mention of patterns can be
seen.
