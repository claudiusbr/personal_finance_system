\section{Introduction} \label{sec:Introduction}

A system could be summarised roughly as a solution to one or more problems. One
of the first steps in order to build this kind of solution is to try to
understand the problem -- that is, try to map the requirements of the software.
Vaasen et al. (2009 \cite[cited][p.~8]{Boczko:2012:IAI:2331376}) suggests that
an accounting information system's main purpose is to provide information to
internal and external stakeholders. Although this refers to accounting systems
for business, it could be argued that the same concept could be applied for
personal finance systems -- except that, in this case, the main stakeholder
would be the individual using the system (that is, the user). In fact, one of
the most widely known accounting systems available in the market,
Quicken\texttrademark, was conceived around the idea that there should be more
efficient and less tedious ways to organise one's personal financial
information than doing it manually (\cite{quicken2017about}). This project has
been developed based on these ideas.

This report documents the work of the project. Each chapter delineates a
specific aspect of the development lifecycle, which is in line with the
development process listed in \ref{sec:DevelopmentMethod}. Chapter
\ref{sec:Requirements} identifies the identified requirements which were used
as motivation for the system to be developed.

The system developed for this project has been modelled after the principle of
\emph{double entry bookkeeping}, from the accounting domain, which states that
``money is never created or destroyed -- it merely moves from one account to
another'' (\cite[][Section 6.2]{fowler1997analysis}). More specifically, double
entry is the principle which ensures that every transaction always affects two
accounts, one being credited (Out) and one debited (In). An account, for the
scope of this project, refers either to a category created by the user, or to
the user's \emph{cash book} -- the contents of their bank account plus any
manual entry which they make. In bookkeeping, each account can be classified as
\emph{asset, liability, income} or \emph{expenditure}. Whether the account
increases or decreases will depend on which of these categories it falls under:
\emph{debits} will increase \emph{assets} and \emph{expense} accounts, and
\emph{credits} will increase \emph{liability, capital} or \emph{income}
acccounts (\cite[][pp.~18-19]{wood2004book}).

Analysis of requirements has been incorporated in the design section (Chapter
\ref{sec:Design}). This is also where the first mention of patterns can be
seen.
