\section{Appendix I: Types of Requirements and the use of UML} \label{appendix1}

Bennett et al.
(\citeyear[][pp.~140-142]{bennett2010object}) categorises requirements as being
of three types:

\begin{description} \item[Functional Requirements]
    The system's functionality -- what it is expected to do.
    
  \item[Non-functional Requirements]
    How well the system delivers its functionality. These requirements are
    related to the performance, scalability, availability, recovery time,
    security, and others.

  \item[Usability Requirements]
    These relate to how effectively, efficiently and satisfactorily users can
    achieve their goals in the existing system. User interfaces can play a big
    part in meeting these requirements.
\end{description}

\subsection{The use of Universal Modelling Language (UML) constructs}
\label{sec:Introduction.methodology.uml}

UML is a modelling language created with the intention of providing system
architects, software engineers and developers with a common set of modelling
tools, with a defined syntax, which would help them better analyse and design
software-based systems, and to model business and similar processes
(\cite[][p.~43]{omg2015uml}). It defines several constructs which have been
employed throughout this report in order to model the specifications of the
system, such as Use Case, Activity, Class, Sequence and Communication Diagrams.  

In the class diagrams used, some thought was given to how to model the
relationships between entities. In the end, it was decided that transitory
relationships should be modelled as dependencies, and more permanent ones would
be modelled as associations.
