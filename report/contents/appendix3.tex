\section{Appendix III: How to Build and Run the Application} \label{appendix3}

\noindent
As this appendix is a list of instructions, some of it will be in an imperative
voice.

\noindent
These instructions are also available in the \texttt{README} of the GitHub
Repository for this project:\\
\href{https://github.com/claudiusbr/personal_finance_system}{https://github.com/claudiusbr/personal\_finance\_system} 

\noindent
A portable version of the app with all the information has been provided in the
accompanying USB memory, in the \texttt{PortableBinaryVersion} folder. In order
to run it, you need to execute the steps on section \ref{appendix3.setupMySql}
of this appendix first, then follow the steps on section
\ref{appendix3.runningcompiled}.


\subsection{What is needed}
\begin{itemize}
  \item
    SBT 1.1.1 (can be downloaded using SDKMAN -- \href{http://sdkman.io/}{http://sdkman.io/})

  \item
    Scala 2.12.4 (can be downloaded with SBT, or using SDKMAN -- 
    \href{http://sdkman.io/}{http://sdkman.io/})

  \item
    MySQL
    \begin{itemize}
      \item
        also, the \texttt{schema.sql} file, which should be included in the
        accompanying USB memory under
        \texttt{.../SourceCode/original\_db\_schemas/mysql/schema.sql}, but
        also can be downloaded from
        \href{https://github.com/claudiusbr/personal_finance_system/tree/master/code/original_db_schemas/mysql}{the
        GitHub repo};

      \item
        and the \texttt{test\_schema.sql}, which should also be included in the
        USB memory under
        \texttt{.../SourceCode/original\_db\_schemas/mysql/test\_schema.sql},
        or should be downloaded from
        \href{https://github.com/claudiusbr/personal_finance_system/tree/master/code/original_db_schemas/mysql}{the
        GitHub repo}.
    \end{itemize}

  \item
    The \texttt{config.properties} file for the main application, which is also
    in the USB under
    \texttt{.../SourceCode/config.properties}, or
    can be downloaded from
    \href{https://github.com/claudiusbr/personal_finance_system/tree/master/code}{the
    GitHub repo};
    \begin{itemize}
      \item
        within the file, the \texttt{MySqlUrl} property value needs to be
        replaced by the URL of the MySql server containing the schemas listed below.
    \end{itemize}


  \item 
    A \texttt{private.properties} file containing the MySQL username and password for
    the user that can access the two database schemas for this project:
    \texttt{personal\_finance\_schema} and \texttt{test\_personal\_finance}. It
    should be a text file containing the two lines below, where
    \texttt{username} and \texttt{password} should be replaced by the username
    and password for the MySql server containing these schemas.

    \begin{lstlisting}[frame=single]
      MySqlUsername=username
      MySqlPassword=password
    \end{lstlisting}

  \item
    (\emph{Needed only If you are compiling from source:}) the
    \texttt{testprops} and \texttt{testtextfile} files, which are required to
    run the tests (they should be in the USB in the\\
    \texttt{.../SourceCode/src/test} folder, or can be downloaded from
    \href{https://github.com/claudiusbr/personal_finance_system/tree/master/code/src/test}{the
    GitHub repo}).


\end{itemize}

\subsection{Building and Running it}
\subsubsection{set up the project}
\begin{enumerate}
  \item
    Copy the source files from the USB's \texttt{SourceCode} folder (or
    download the files from the main GitHub repo's \texttt{code} folder) into
    an empty folder, then \texttt{cd} into it (make sure it is the one
    containing the \texttt{build.sbt} file). This is now the root folder for
    the code;
    \begin{itemize}
    \item
      make sure the \texttt{testprops} and \texttt{testtextfile} files are in the
      \texttt{src/test} directory;
    \end{itemize}

  \item
    create the \texttt{private.properties} file mentioned above, and save it in
    the root folder;
    \begin{itemize}
      \item
      add your username and password to it, if you haven't yet;
    \end{itemize}

  \item
    make sure the \texttt{config.properties} is also in the root folder;

\end{enumerate}


\subsubsection{Set up MySql} \label{appendix3.setupMySql}
\begin{enumerate}
  \item
    create a database schema called \texttt{personal\_finance\_system};

  \item
    create a database schema called \texttt{test\_personal\_finance};

  \item
    give the user on \texttt{private.properties} access to writing to them;

  \item
    Load the schemas into mysql

    \begin{lstlisting}
    $ mysql -u username -p personal_finance_system < schema.sql
    $ mysql -u username -p test_personal_finance < test_schema.sql
    \end{lstlisting}
\end{enumerate}


\subsubsection{Run SBT from the root folder}
\begin{lstlisting}
$ cd personal_finance_system/code
$ sbt run
\end{lstlisting}


\subsection{Running the compiled version instead} \label{appendix3.runningcompiled}
A pre-assembled executable \texttt{.jar} file is available on the USB memory.
To run it instead of compiling the code, follow the steps from
\ref{appendix3.setupMySql} to set up the database schemas, save the
\texttt{config.properties} and \texttt{private.properties} files in the folder
containing the \texttt{PersonalFinanceSystem.jar} file (make sure you enter the
username and password in the private.properties file), and execute the
\texttt{.jar} to run the application:
\begin{lstlisting}
$ java -jar PersonalFinanceSystem.jar
\end{lstlisting}

The current iteration does not have any ways of communicating with the user
yet, so if running this version and something goes wrong the system will not
tell you what it is, so the recommended approach is to build it from source.

\subsection{How to use it}
Make sure your bank statements are in a csv format, and that the column names
match the below exactly (non-case sensitive, though):
\begin{lstlisting}
date,description,amount
... , ... , ...
...
\end{lstlisting}

As this tends to be the trend with bank statements (or at least mine), all
income should be negative, and all expenditure positive, so if you are
uploading either bank or credit card statements, make sure all income and
expenditure is exactly with these signs.
