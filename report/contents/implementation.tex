\section{Implementation} \label{sec:Implementation}

\textbf{TODO: Read the lightbend guide https://www.lightbend.com/lagom-framework to Scala and Microservices}

\textbf{TODO: see if the below still makes sense}
The constraints from \texttt{Transaction} and \texttt{Pattern} were implemented
in the business logic code as follows:
{
  \small
  \lstinputlisting[
    language=Scala,
    firstline=15,
    lastline=24,
    caption={
      method to validate \texttt{Transaction} entries
    },
    label=DescriptiveLabel
  ]{../code/src/main/scala/businesslogic/transaction/Transaction.scala}
}

{
  \small
  \lstinputlisting[
    language=Scala,
    firstline=4,
    lastline=16,
    caption={
      snippet of \texttt{Pattern} showing requirement for it to have at least one Category
    }
  ]{../code/src/main/scala/businesslogic/transaction/Patterns.scala}
}

As Fowler (\citeyear[][]{fowler1997analysis}), one of the classes should be
responsible for keeping track of the relationship. As seen above, it was
decided that each \texttt{Category} will keep track of its patterns, and this
will at the same time enforce the constraint at the application layer level.


\subsection{Scala Case Classes} \label{sec:Reflections.ScalaCaseClasses}
One of the benefits of Scala are its case classes. They are very useful for
pattern matching, but also come with a few perks such as the \texttt{copy()}
method. This method allows for the class to be copied with some or all its
members modified. It could be said that it is a language-native implementation
of the \emph{Prototype Design Pattern}
(\cite[][location~2461]{nikolov2016scala}), and throughout the implementation
and testing stage it proved to be a useful feature.

Its utilisation can be seen in the \texttt{Transaction} class, as one of the
tools used to add entries to a \texttt{Category} without having to change the
state of a specific instance -- more similar to what is done in the
\emph{Functional Programming} paradigm:
{
  \small
  \lstinputlisting[
    language=Scala,
    firstline=32,
    lastline=38,
    caption={
      extract of the \texttt{Transaction} class showing the \texttt{copy()}
      method in action
    }
  ]{../code/src/main/scala/businesslogic/transaction/Transaction.scala}
}

